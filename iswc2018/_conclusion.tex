\section{Conclusion and Future Work} \label{sec:conclusion}
We have presented the DOREMUS resource---a collection of linked RDF datasets representing the catalogs of music works of three major French cultural institutions. The construction of this resource implies the implementation of a processing pipeline that allows for the conversion of the original data to RDF following the DOREMUS ontology, the development, SKOS-ification and alignment of a number of music-specific vocabularies and the interlinking of the datasets, which results in the construction of a reference pivot graph of musical works shared by or unique to the three  institutions. This pipeline defines the data production paradigm of DOREMUS that is applicable to other music-library data---the described process is deterministic, extensible, reproducible and documented in numerous pedagogical materials published online. A number of tools acting at different layers of this pipeline have been introduced: the {\tt marc2rdf} data converter, the {\it Legato} data linking system, the web-interface for SKOS thesauri alignment and mapping validation and enrichment YAM++ {\it on line},  as well as the exploratory search engine \textsc{Overture}. We have relied on existing tools where appropriate (matching strings to URI), but the heterogeneity of the input data and the specificity of the librarian practices made this impossible in many cases. In terms of datasets, DOREMUS currently has published (1) three RDF graphs of musical works coming from the BnF, the Philharmonie de Paris and Radio France, (2) a pivot graph currently containing the certain links established automatically between the graphs of musical works, together with the results of the pairwise linking of these graphs, (3) expert curated benchmarks for evaluation of data linking systems, (4) a rich set of music-specific SKOS vocabularies together with their alignments.

We are currently in the process of applying the data conversion and linking workflow to two additional databases from Radio France. Natural Language Processing techniques are being included in the conversion process in order to parse the numerous free-text fields. \textsc{Overture} will soon host all the links between the interlinked works, giving access at the same time to the joined knowledge and to the different information provenances. We have developed a web interface to assist the process of manual validation of links reducing the human effort, which is currently being deployed online.  Alignments of our data to established datasets (in particular MusicBrainz) are currently being generated.