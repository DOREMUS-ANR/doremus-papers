\section{Introduction} \label{sec:intro}
The Linked Open Data (LOD) paradigm for data representation, sharing and publishing has been more and more appealing to the world of museums and libraries over the past years. The LOD project and the semantic web in general offer technological means for data reuse, increased visibility and data sharing on the web, data federation and facilitated exchange of metadata by the creation of links across resources. Attracted by these possibilities, many major actors from the library world, such as the Library of Congress (LOC) or the French National Library (BnF), have embraced semantic web technologies with the goal to open their archives and catalogs to the web. This process has resulted in a number of openly available and explorable RDF graphs reflecting the rich content of numerous libraries and cultural institutions from all over the world~\cite{marden2013linked}.

The DOREMUS project follows this line of research and practice, with a particular interest in classical and traditional music, so far relatively underrepresented on the LOD.\footnote{\url{http://www.doremus.org}} Three major French cultural institutions---the BnF, Radio France (RF) and the Philharmonie de Paris (PP)---have joint efforts with data and social science academics in order to develop shared methods to describe semantically their catalogs of music works and events and open them to the web community. A major contribution of the project is the development of the DOREMUS ontology\footnote{\url{http://data.doremus.org/ontology/}} which extends the well-known CIDOC-CRM and FRBRoo models for representing bibliographic information\footnote{\url{http://new.cidoc-crm.org/frbroo/}}, adapting it to the domain of music, thus filling an important representational gap. A number of shared vocabularies about music-specific concepts (such as musical genres or keys) have been collected or developed, linked and published using the SKOS standard. The data from the catalogs of the three partner institutions comes in MARC or XML  formats. Specific tools for data conversion to RDF following the DOREMUS model have been developed. This process results in the construction of several knowledge graphs about music works and events, which have been linked using a specifically developed for this purpose data linking tool. For evaluation purposes, a benchmark has been created manually by the library experts and shared to the semantic web community as part of the Ontology Alignment Evaluation Initiative (OAEI). The data fusion process results in the construction of a pivot graph of shared and unique musical works. Finally, an exploratory search engine is developed that allows to browse the knowledge graphs.

This paper covers the components of the DOREMUS workflow described above, which altogether form a paradigm for lifting, linking and publishing music library metadata. We present in detail the DOREMUS knowledge graphs with a focus on the (re-)used models and vocabularies and the processes that allow for their (re-)production and fusion. The contributions of this work are:

$\bullet$ A model for describing musical works and events extending FRBRoo together with a number of shared and linked music-specific controlled vocabularies.

$\bullet$ Three knowledge graphs about music works that represent the catalogs of three major French cultural institutions.

$\bullet$ An approach to interlink these graphs resulting in the construction of a pivot graph, containing all unique works and links to the original graphs. 

$\bullet$ A set of benchmark datasets for data linking evaluation.

$\bullet$ A set of tools for data generation, vocabulary alignment and validation, data linking, pivot graph construction, and data search and exploration.

The remainder of the paper is structured as follows. In the next section, we provide general information about the graphs, their form and content, the (re-) used ontologies and controlled vocabularies, as well as statistics. In Section~\ref{sec:develop}, we detail the different components of the DOREMUS data production pipeline, and in particular, the data conversion and linking approaches. In Section~\ref{sec:use}, we demonstrate how this resource has already been used and we discuss its wider expected impact. We present related initiatives in Section~\ref{sec:relwork} before we conclude and discuss future work in Section \ref{sec:conclusion}.