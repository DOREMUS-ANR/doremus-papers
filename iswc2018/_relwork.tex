\section{Related Work and Graphs} \label{sec:relwork}
There has been a significant effort in the last years to open and publish data from the field of cultural heritage \cite{dijkshoorn2014rijksmuseum}. An overview of related projects is given in \cite{marden2013linked}, where the authors provide an evaluation of the various initiatives with regard to the well-known five-stars open data  rating, applied to the cultural domain.%\footnote{\url{https://www.w3.org/DesignIssues/LinkedData.html}} 

Regarding the more specific problem of producing linked data out of library records, addressed by the DOREMUS project, a number of related initiatives have recently been introduced. We refer to the multiple contributions of the Europeana project,\footnote{\url{http://www.europeana.eu}}, unifying and making accessible the catalogs of  numerous libraries, museums and archives across Europe. One of the early efforts in that respect is made by the Library of congress,\footnote{\url{http://id.loc.gov}} which has become a dataset of reference in the field. In the same spirit, related projects include the German National Library linked data service,\footnote{\url{http://www.dnb.de/EN/lds}} the British National Bibliography Linked Data Platform,\footnote{\url{http://bnb.data.bl.uk/docs}} the open data project of the French National Library BnF\footnote{\url{http://data.bnf.fr}} or, more recently, the Virtual Library  Miguel de Cervantes project \cite{candela2017migration}.

In the majority of the cases, data comes  in a given MARC variant and has to be converted to RDF. In certain cases the migration process goes through an intermediate phase of translation to relational database \cite{candela2017migration}, or data is being directly converted to RDF based on the standards of bibliographical description, such as FRBR. DOREMUS follows this line of work by implementing its own expert-defined mappings-based conversion mechanism, enriching FRBRoo with more than 40 classes and 100 properties. The resulting (DOREMUS) model fills the important gap between library content description and music metadata. 

As compared to music-related datasets, we outline that the BBC open datasets have tracks only, the Dutch Library (part of Europeana) has only publications, CPDL\footnote{\url{https://www.cpdl.org/}} is specialized for chorus (with scores and midi), while DOREMUS is general and can glue these datasets. MusicBrainz \cite{musicbrainz}, one of the most popular knowledge bases about music metadata, started a few years ago its process of exposing its data as semantic triples through the platform LinkedBrainz \cite{linkedbrainz}. In contrast to DOREMUS, which follows a librarian structure, MusicBrainz follows a more commercial practice giving a central role to tracks, albums and artists (un-distinguishing the composer from the performer), at the expense of all the information connected to the work concept (genre, casting, key, etc).