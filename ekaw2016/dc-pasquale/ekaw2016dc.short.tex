%%%%%%%%%%%%%%%%%%%%%%%%%%%%%%%%%%%%%%%%%%%%%%%%%%%%%%%%%%%%%%%%%%%%%%%%%%%%%%%%%%%%%
%%%  EKAW 2016 DC - Modeling, Exploring and Recommending Music in its Complexity  %%%
%%%%%%%%%%%%%%%%%%%%%%%%%%%%%%%%%%%%%%%%%%%%%%%%%%%%%%%%%%%%%%%%%%%%%%%%%%%%%%%%%%%%%

\documentclass{llncs}

\usepackage{url}
\usepackage{graphicx}
\usepackage{xtab}
\usepackage[inline]{enumitem}
\usepackage[title]{appendix}
\DeclareGraphicsExtensions{.png}

%%%%%%%%%%%%%%%%%%%%%%%%%%%%%%%
%%%  Beginning of document  %%%
%%%%%%%%%%%%%%%%%%%%%%%%%%%%%%%

\begin{document}

\title{Modeling, Exploring and Recommending\\ Music in its Complexity}

\author{Pasquale Lisena}
\authorrunning{Lisena}
\institute{EURECOM, Sophia Antipolis, France \\
\email{pasquale.lisena@eurecom.fr}}

\maketitle

%%%%%%%%%%%%%%%%%%
%%%  Abstract  %%%
%%%%%%%%%%%%%%%%%%

\begin{abstract}
Knowledge models that are currently in-use for describing music metadata are insufficient to express the wealth of complex information about creative works, performances, publications, authors and performers. In this thesis, we aim to propose a method for structuring the music information coming from heterogeneous librarian repositories. In particular, we research and design an appropriate music ontology based on existing models and controlled vocabularies and we implement tools for converting and visualizing the metadata. Moreover, we research how this data can be consumed by end-users, through the development of a web application for exploring the data. We ultimately aim to develop a recommendation system that takes advantage of the richness of the data.

\keywords{Ontology, FRBRoo, Music Metadata, Schema.org, Recommender System}
\end{abstract}

%%%%%%%%%%%%%%%%%%%%%%%%%%%
%%%  Problem Statement  %%%
%%%%%%%%%%%%%%%%%%%%%%%%%%%

\section{Problem Statement}
\label{sec:problem}
Music metadata can be very complex. Metadata describing a well-known masterpiece such as the \textit{Moonlight Sonata} can describe its composition by Beethoven, its scores in the handmade or printed version, the interpretations by pianists, the orchestrations and arrangements. Performances, recordings, music albums can also be described. Numerous actors are involved in this media production chain: composers, performers with different roles, conductors, etc.

Online musical content holders offer a very simplified version of this information, focused on the track as the atomic unit, the artist as the unique carrier of the authorship, and presence in the same album as unique possible relationship between tracks. While a simplified version of the metadata is enough for commercial purpose, the whole complexity of the music information opens up new possibilities for advanced search, visualization, and recommendation strategies.

Libraries, in contrast, have more structured information, often represented in specialized formats such as MARC\footnote{\url{https://www.loc.gov/marc/}}. The limit of this approach is that developers are tied to these formats with non-explicit semantics. Moreover, each institution hosts data in its repositories, often using a particular dialect of MARC. As a result, the music metadata has currently no chances of reconciliation and interconnection. The benefits about moving from MARC to an RDF-based solution consist in the interoperability and the integration among libraries and with third part actors, with the possibility of realizing smart federated search~\cite{byrne2010strongest}.

This thesis aims to propose a new model for representing music metadata in its full complexity and design, implement and evaluate novel applications that use this metadata with the purpose of exploring and recommending music.

This research is being developed in the context of the DOREMUS project\footnote{\url{http://www.doremus.org}}~\cite{achichidoremus}, in which three leading cultural institutes in France --- the BnF (Biblioth\`eque Nationale de France), the Philharmonie de Paris and Radio France --- join forces with companies and academic institutions in order to make the music knowledge in their catalogs available and re-usable on the web of data.

%%%%%%%%%%%%%%%%%%%%%%%%%%%%
%%%  Proposed Approach   %%%
%%%%%%%%%%%%%%%%%%%%%%%%%%%%

\section{Proposed Approach}
\label{sec:approach}
Three different challenges represent the ambitious goals of this research thesis.

The first challenge is to find an appropriate ontology model for capturing the richness of music information, taking into account all its components. The data in MARC format from the source institutions will be converted independently in RDF. After that, a reconciliation process should be realized. \texttt{sameAs} links should be found on the resources from the different institutions that describe the same work and let them converge in a unique (virtual) graph containing the whole information at our disposal. This should be realized by identifying the features that enable to disambiguate resources. Controlled vocabularies must be used for describing features as keys, genres, etc., and disambiguating their values.

The second challenge consists in providing a simplified version of this structured metadata, tailored to be consumed by search engines and third-part applications. Because of its central role in the web, we will provide mappings of this ontology into Schema.org. Some research questions are:
\textit{1. which information should be presented in this simplified version and which one is considered too complex?}
\textit{2. which methods should be used for realizing this simplification?}

Finally, our aim is to demonstrate the benefits that this data can produce once being consumed and displayed to the end-user. We consider taking the user into account as a crucial requirement for the design of these systems. In this context, an application will be developed that consists in two parts. First, an exploratory search engine for musical data will be designed and developed.
\textit{1. Can the knowledge model simplification operated in the second challenge be used to improve the user experience?} 
\textit{2. How complex concepts and relationship should be displayed to the end-user?}

On top of the exploratory interface, we will build a recommendation engine.
\textit{1. Are the recommendation systems currently in-use for music still valid with this complexity?}
\textit{2. Is the rich model better than the simplified (Schema.org-based) one for feeding the recommendation?}

%%%%%%%%%%%%%%%%%%%%%%%%%
%%% State of the Art  %%%
%%%%%%%%%%%%%%%%%%%%%%%%%

\section{State of the art}
\label{sec:state-art}

\subsubsection*{Musical data representation}
The description of music is historically connected to catalog information models, among which FRBR is one of the most popular. FRBR and CIDOC-CRM, an ontology for describing museum information, have been harmonized in the FRBRoo ontology for describing arts~\cite{doerr2008frbroo}. This model considers a cultural entity as the combination of the Work that is realised into an Expression through an Event creation. Among music specific ontologies, the most popular is Music Ontology~\cite{raimond2007music}. The experiences about converting data from MARC to RDF include MARiMbA~\cite{greenberg2013datos}, that uses the FRBR model and manages also  the interlinking, loading and visualization of the data.

\subsubsection*{Music recommendation}
Music recommendation is an active and popular research field. Most of the important actors in online music provide some sort of recommendations, often making use of collaborative recommendation algorithms. The need to discover novelty in results requires the use of approaches like the Long Tail curve for also including less popular works in recommendation~\cite{celma2009music}. A feature-combination hybrid approach is presented in~\cite{ostuni2015soundrec}: music datasets are semantically enriched through external graphs and combined with collaborative features for providing better results in terms of both accuracy and novelty. The approaches completely based on the knowledge of the model are less common, like a recommendation system driven by the nearby places of interests~\cite{kaminskas2012knowledge}.

%%%%%%%%%%%%%%%%%%%%%%
%%%  Methodology   %%%
%%%%%%%%%%%%%%%%%%%%%%

\section{Methodology}
\label{sec:methodology}
Different parts of this thesis are currently carried out in parallel, in order to have in mind the final goals during the development of the various strategies.

\subsubsection*{DOREMUS Ontology.}
The DOREMUS ontology\footnote{\url{http://data.doremus.org/ontology}} is being developed as an extension of the FRBRoo model. This choice comes from two reasons: a fine granularity of description granted by the triplet pattern, and the possibility to interoperate with systems dealing with any kind of cultural data. A modeling group of experts in music cataloging and knowledge representation, is currently working on the ontology, that currently defines 46 classes and 144 properties.

\subsubsection*{Controlled vocabularies.}
Different kind of vocabularies are needed for describing music: some are already available on the web (like medium of performance or musical genres), while others do not exist (like the types of derivation). Not all of them are published in a suitable format for the Web of Data, and there is often little interconnection between vocabularies. Our effort is to publish SKOS representation for each controlled vocabulary, with appropriate relationships and mappings between the different sources.

\subsubsection*{Simplification through Schema.org.}
Schema.org contains different types for describing music: CreativeWork, MusicComposition, MusicRecording, MusicGroup, etc. Passing from FRBRoo to the Schema.org means mapping the concept expressed in a complex ontology to a simpler one. We have proposed a method composed of a series of recipes that enables to perform this operation based on the observation of the graph and the individuation of the relative Schema.org types through similarity criteria~\cite{lisena2016mapping}.

\subsubsection*{Recommendation.}
Our starting point for the study of a recommendation algorithm consists in two human-made collections: playlists from Radio France web radios and concert programs from Philharmonie of Paris, realized by an editorial action made by experts and aiming at discovering novel works. In parallel, we will consider the leading music providers: we are developing ARyTREx\footnote{\url{https://github.com/fernanev/ARyTREx}}, a web application that aims to simulate the recommendations made by Spotify and Last.fm given a seed artist or track and using their respective APIs. The application enables to collect recommendation paths that can feed a semi-supervised machine learning algorithm. One goal will be to identify which features are the most important in the recommendation. Programs, playlists and third-parts recommendation results will play a key role in the evaluation of the algorithm, that can be iteratively corrected by the comparison between its output and these collections. Precision and recall metrics will be used to measure this accuracy.

%%%%%%%%%%%%%%%%%%%%%%%%%%%%%%
%%%  Preliminary results   %%%
%%%%%%%%%%%%%%%%%%%%%%%%%%%%%%

\section{Preliminary results}
\label{sec:results}
We have already developed two prototypes, both as open source software~\cite{lisena2016exploring}.

\textsc{marc2rdf}\footnote{\url{https://github.com/DOREMUS-ANR/marc2rdf}} is a prototype for converting the bibliographic records from UNIMARC and INTERMARC to an RDF graph following the DOREMUS model. The conversion relies on explicit expert-defined transfer rules (or mappings), that define for each property/class in the DOREMUS ontology where to find the relative information in the MARC file. The software contains also a \textit{string2uri} component, that performs an automatic mapping of string literals to URIs coming from controlled vocabularies.

A prototype of an exploratory search engine named \textsc{Overture}\footnote{\url{https://github.com/DOREMUS-ANR/overture}} is under development. The challenge is in giving to the final user a complete vision on the data of each work, performance, score, recording and letting him/her to understand how they are connected to each other. The exploratory interface is available at \url{http://overture.doremus.org}.

%%%%%%%%%%%%%%%%%%%%%%%%%%%%%%%%%%%%
%%%  Conclusion and Future Work  %%%
%%%%%%%%%%%%%%%%%%%%%%%%%%%%%%%%%%%%

\section{Conclusion and Future Work}
\label{sec:conclusion}
Representing the information about classical music is a complex activity, that involves different sub tasks. During this research, we want to express the musical metadata from their current MARC format in a suitable ontology that can preserve their complexity. From that, a simplification through Schema.org will guide the realization of an exploratory interface, while the richness of the model will fully support the development of a recommender system.

We are publishing a large DOREMUS dataset composed of the ontology, controlled vocabularies and data about classical and jazz works, accessible at \url{data.doremus.org/sparql}. Disambiguation of data and interlinking between resources should be performed in the context of controlled vocabularies. The data coming from the conversion of partner institutions will be interlinked, so that different descriptions of the same work go enriching each other. Then, this research will go deep in the data presentation and recommendation.

\subsubsection*{Acknowledgments}
I would like to thank my supervisor Rapha\"el Troncy for his ongoing support in. This work has been partially supported by the French National Research Agency (ANR) within the DOREMUS Project, under grant number ANR-14-CE24-0020.

% ---- Bibliography ----
\bibliographystyle{abbrv}
\bibliography{bib-doremus}

\end{document} 